\documentclass[11pt,a4wide]{article}
\begin{document}

\begin{titlepage}

\newcommand{\HRule}{\rule{\linewidth}{0.5mm}} % Defines a new command for the horizontal lines, change thickness here

\center % Center everything on the page
 
%----------------------------------------------------------------------------------------
%	HEADING SECTIONS
%----------------------------------------------------------------------------------------

\textsc{\LARGE Cobourg Collegiate Institute }\\[1.5cm] % Name of your university/college
\textsc{\Large ENG 4UI}\\[0.5cm] % Major heading such as course name

%----------------------------------------------------------------------------------------
%	TITLE SECTION
%----------------------------------------------------------------------------------------

\HRule \\[0.4cm]
{ \huge \bfseries Religious Symbolism in Marquez's \emph{Chronicle of a Death Foretold}\\[0.4cm] % Title of your document
\HRule \\[1.5cm]
 
%----------------------------------------------------------------------------------------
%	AUTHOR SECTION
%----------------------------------------------------------------------------------------

\begin{minipage}{0.4\textwidth}
\begin{flushleft} \large
\emph{Author:}\\
Peter \textsc{Kowalchuk-Reid} % Your name
 
\end{flushleft}
\end{minipage}
~
\begin{minipage}{0.4\textwidth}
\begin{flushright} \large
\emph{Supervisor:} \\
Ms. C.  \textsc{Stone} % Supervisor's Name
\end{flushright}
\end{minipage}\\[4cm]

% If you don't want a supervisor, uncomment the two lines below and remove the section above
%\Large \emph{Author:}\\
%John \textsc{Smith}\\[3cm] % Your name

%----------------------------------------------------------------------------------------
%	DATE SECTION
%----------------------------------------------------------------------------------------

{\large \today}\\[3cm] % Date, change the \today to a set date if you want to be precise

%----------------------------------------------------------------------------------------
%	LOGO SECTION
%----------------------------------------------------------------------------------------

%\includegraphics{Logo}\\[1cm] % Include a department/university logo - this will require the graphicx package
 
%----------------------------------------------------------------------------------------

\vfill % Fill the rest of the page with whitespace

\end{titlepage}

 
\paragraph{} In \emph{Chronicle of a death foretold} Gabriel Garcia Marquez 
relies on his reader's shared understanding of Christian and biblical themes.  He creates biblical narratives, and then uses them for irony by ending these 
narratives in unexpected---and often disappointing--- ways. He uses 
Christianity's two most widely known stories as the basis for the novel: the 
story of Genesis and the Passion of Christ. These two sections of the bible both 
offer a view into Colombian society at the time. By using the Bible as a basis 
for the novel, he is able to play upon reader's shared expectations. He uses the 
irony of biblical figures forced to deal with earthly problems as a way to 
criticise both the society he is writing about, and the religion that controls 
it.

%Life similarities
\paragraph{}The portrayal of Santiago Nasar as an analogue of Jesus of Nazareth 
is one of the novel's key images. Marquez creates similarities between the two 
men that are obvious, the most striking being that Santiago Nasar and Jesus both 
wear all white; a traditional sign of purity. The two men share the common 
cultural heritage of all Semites as ``people of the book''\cite[3:113]{quran}.  
There are  multiple allusions to Christ's passion in Santiago Nasar's murder.
When Pablo and Pedro Vicario kill Santiago Nasar, they choose to use knives,
this is significant because in Christianity the piercing of the body of Christ
plays a major symbolic role. The placement of stab wounds through the hands,
and that Santiago Nasar's hand was pinned to the wooden door all evoke images
of Jesus being nailed through his hands to the cross. The detachment of the
brothers echoes the  detachment of the romans as they killed Jesus. 

%Death similarities
\paragraph{}Both the passions of Christ and of Santiago Nasar contain miracles, 
for Jesus it was water pouring from the wound in his side, for Santiago Nasar it 
is that nothing at all would come out of his many wounds. All the time the 
brothers are stabbing him, Santiago Nasar remains composed, and silent: ``He 
didn't cry out again'' echoes Jesus' silence on the cross ``he opened not his 
mouth''\cite[Isaih 53:7]{bible}. After Santiago Nasar is stabbed, despite being 
riddled with wounds and holding in his own guts, he rises from the ground and
is able to walk calmly around his house,  speak to the narrator's aunt and go
in through the back door before collapsing. This is  Santiago Nasar's 
resurrection, he should have died long before , but he instead rises from the 
ground and walks calmly ``with his usual good bearing, measuring his steps 
well,'' \cite[pg.~120]{chronicle} After his resurrection he first speaks to
a woman: Wenefrida Marquez, this is analogous to Jesus' first appearing to Mary 
Magdalene. By making Santiago Nasar so closely resemble Jesus, Marquez gives us 
certain expectations for how his character should be treated. In a town so 
deeply Christian as Nasar's, we expect his treatment to be just and 
compassionate, as Christ taught.

%Burial dissimilarities
\paragraph{} After his death, the similarities between Jesus and Santiago Nasar 
are broken. When Jesus was killed his body was carefully prepared for burial by 
being anointed with herbs and wrapped in linen. The way Santiago Nasar's body is 
prepared comes as a surprise, we expect him to have a burial similar to that of 
Christ, but instead he is left in the care of an incompetent priest and student 
to be badly autopsied and stuffed with lime, left to sit in a hot room while the 
dogs beg for his guts. Marquez creates deep irony by showing the novel's Christ 
figure in such an undignified burial. Through this irony, the villagers are 
exposed as the hypocrites that they are, the godly people, who are willing to do 
anything for the Bishop, do nothing to stop the crucifixion of one of their own. 
Pablo and Pedro Vicario even go as far as to tell father Amador that they are 
innocent ``Before God and before men'',  the irony being that they killed a man 
representing God in the form of a man, and yet still claim they are innocent 
before both.  

% I need to marry the ideas of Bayardo's imitation of God and his role as Adam
% I need to address this directly I must say it word for word.

%%%%%%%%%%%%%%%%%%%%%%%---COMPARISONS---%%%%%%%%%%%%%%%%%%%%%%%%%%%%%%%%%%%%%%
\paragraph{}After their wedding, Marquez establishes Angela Vicario and Bayardo 
San Roman in a narrative that parallels the story of Adam an Eve. Angela mirrors
Eve, Bayardo in turn mirrors Adam. Their house in a lush tropical setting is 
meant to evoke images of Eden. While Marquez strongly parallels Genesis,
he does not replicate it. Unlike in the bible, --- where God is ever-present
--- in \emph{Chronicle of a death Foretold}, God is nowhere to be found. God 
isn't there to pass judgement or offer guidance. In his absence, Bayardo tries
to act as God, using his wealth and influence in imitation of God's power.
For the most part his imitation is successful,
the villagers come to believe him capable of anything.  
\cite[pg.~27]{chronicle}\\


In both couples the woman was brought into the story to please a man. Eve is
fashioned of Adam's rib. Bayardo finds Angela Vicario, and buys her from her
family. He chose a girl who was raised in a sheltered house. He wanted a 
sheltered wife because she will be easier to control. With no greater perspective
to draw on, she is unable to realise the injustice of her oppression. \cite[pg.~34]{chronicle} 

%%%%%%%%%%%%%%%%%%%%%%---CONTROL-OF-WOMEN---%%%%%%%%%%%%%%%%%%%%%%%%%%%%%%%%%%%%
\paragraph{}In Genesis and \emph{Chronicle of A Death Foretold} the God figure
uses control of knowledge to place the woman beneath him. Both Eve and Angela
are kept ignorant, so as to limit their power. Angela is controlled through
her sexuality, never allowed to learn the reality of sex, only ever
being told old wives' tales. When Angela Vicario chooses to reveal her lost 
virginity to Bayardo, she is taking that power from him. She is choosing to
assume the place of an equal. God is by nature jealous 
\cite[Exodus 20:50]{bible} and will accept none as his equal. Just as God would
not accept Eve as an equal, Bayardo cannot accept Angela as his.
Colombian society taught him he should dominate his wife, and he believes it. 
This misguided belief causes him to destroy both his and Angela's lives. 
The bible taught that a man and his wife should be two parts of one being.
Eve was fashioned Adam's side, she was created as his ``companion'', his equal;
not as his slave. Bayardo. Bayardo's delusions of grandeur --- delusions created
by men in the catholic church --- robbed him of his chance at happiness.

\paragraph{} Angela Vicario's life after her expulsion is reminiscent of the 
sentence given to Eve by God ``and thy desire shall be to thy husband, and he 
shall rule over thee.\cite[Genesis 3:16]{bible}'' Angela Vicario is ruled by 
desire for Bayardo San Roman, even after many years of separation she still 
writes to him with slavish devotion. These letters go unread ans unanswered. 
These letters are prayers to a god who has forsaken her.\\
Where her sentence breaks from the biblical narrative is when Adam and Eve
are cast out, they are cast out together. When Angela Vicario is cast out, she 
is abandoned by her husband. With neither God nor husband to answer her prayers,
she is totally abandoned for wanting her god-given place next to her husband as
his equal

\paragraph{} Angela Vicario's experience shows the influence of liberation and feminist 
theology that was then beginning to hold sway in south America. \cite{lib-theo} The novel
shows how these ideas, although present in the bible, were to conflict
with church doctrine of the day. Marquez's retelling of the story of the garden
questions the very foundation of the catholic church, namely patriarchy.
Marquez's retelling and interpretation lays bare the injustice of controlling
women through the church.

% Conclusion
\paragraph{} Marquez lived in Columbia under a repressive social order.
The way he contrasts biblical narratives with worldly problems leads the 
reader to reconsider the common interpretation of the bible.
Marquez Begins his biblical retellings staying close to the bible,
moving away from the original as hischaracters are faced with 
injustice. The effect can be jarring. The novel's heroes are rewarded 
with the exalted statues of their biblical counterparts


\begin{thebibliography}{99}

\bibitem{chronicle}
Marquez, Gabriel Garcia,
Chronicle of a Death Foretold.
Trans. Alfred A. Knopf,
New York,
Random House, 1983
Print.

\bibitem{bible}
Hebrew - English Bible According to the Masoretic Text,
Mechon Mamre,
2 February 2014,
Web,
30 October 2014.

\bibitem{lib-theo}
Berryman, Phillip,
Liberation Theology,
Philadelphia,
Temple University Press,
1987,
Print.

\bibitem{quran}
Al-Quran,
Al-Quran.info.
Tinghoejvej,
January 2013,
Web,
31 October 2014.


\end{thebibliography}
\end{document}
@
